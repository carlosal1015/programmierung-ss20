\begin{exercisePage}[Netze und normierte Räume][11.5/15]
	
	\begin{exercise}
		Es seien $X$ ein topologischer Raum, $M \subset X$ und $u \in X$. Beweisen Sie: 
		\begin{enumerate}
			\item Es gilt genau dann $u \in \cl(M)$, wenn ein Netz $\folge{u_\alpha}{\alpha \in I} \subset M$ gegen $u$ konvergiert.
			\item $M$ ist genau dann kompakt, wenn $M$ Netz-folgenkompakt ist.
		\end{enumerate}
		Es seien zudem $Y$ ein weiterer topologischer Raum und $\abb{F}{X}{Y}$. Beweisen Sie:
		\begin{enumerate}[start=3]
			\item F ist genau dann stetig, wenn $F$ Netz-folgenstetig ist.
		\end{enumerate}
		\textbf{Hinweis:} Hat ein Netz $\folge{x_\alpha}{\alpha \in I} \subset X$ kein konvergentes Teilnetz, so hat es auch keinen Häufungspunkt. Das heißt, zu jedem $x \in X$ existiert eine Umgebung $U$ von $x$ und ein $\alpha \in I$ mit der Eigenschaft, dass für $\beta \in I$ mit $\alpha \le \beta$ stets $x_\beta \notin U$ gilt. Vergleiche dazu auch den Hinweis zu Aufgabe 6.
	\end{exercise}

	\begin{enumerate}[leftmargin=\zulength, label=(zu \alph*)]
		\item \begin{proof-equivalence}
			\hinrichtung Bezeichne mit $I = \mathcal{U}(u)$ die Menge der Umgebungen eines Punktes $u \in X$ und betrachte die Richtung $U \preccurlyeq V \defequiv V \subseteq U$. Definieren wir nun ein Netz $\abb{u}{I}{M}$ durch Auswahl eines $u_\alpha \in \alpha \cap M$ für alle $\alpha \in I$. Dieses existiert aufgrund des Auswahlaxioms und der Tatsache, dass für $u \in \cl(M)$ gilt $\alpha \cap M \neq \emptyset$ für alle $\alpha \in I$. Nun konvergiert dieses Netz bereits nach Konstruktion: Sei $U$ eine offene Umgebung von $u$, d.h. insbesondere $u \in U$. Nach Definition gilt dann $U \in \mathcal{U}(u)$ und wenn $U \preccurlyeq \alpha$ gilt, dann ist $u_\alpha \in \alpha \subseteq U$, d.h. alle Netzglieder beginnend bei $U$ liegen in $U$. 
			\rueckrichtung Sei $\folge{u_\alpha}{\alpha \in I} \subseteq M$ ein gegen $u$ konvergentes Netz. Angenommen $u \notin \cl(M)$, d.h. $u \in X \setminus \cl(M)$, was offen ist. Da $u_\alpha \to u$ existiert ein $\alpha_0$, sodass für alle $\alpha \succcurlyeq \alpha_0$ gilt $u_\alpha X \setminus \cl(M) \subset X \setminus M$ im Widerspruch zum Definitionsbereich des Netzes $\folge{u_\alpha}{\alpha \in I}$.
		\end{proof-equivalence}
	
		\item \begin{proof-equivalence}
			\hinrichtung Angenommen $M$ sei kompakt und es existiere ein Netz $\folge{x_\alpha}{\alpha \in I}$ ohne Häufungspunkt. Dementsprechend existiert auch kein konvergentes Teilnetz. Dadurch, dass kein $x \in M$ ein Häufungspunkt von $\folge{x_\alpha}{\alpha \in I}$ ist, existiert für alle $x \in M$ eine offene Umgebung $U(x) \subseteq M$ von $u$ und ein Index $\alpha_x \in I$, sodass $x_\alpha \notin U(x)$ für alle $\alpha \succcurlyeq \alpha_x$. Dann ist jedoch $\folge{U(x)}{x \in M}$ eine offene Überdeckung von $M$, die aufgrund der Kompaktheit eine endliche Teilüberdeckung hat, d.h. es existieren $x_1, \dots, x_n \in M$, sodass $M \subseteq \bigcup_{i = 1}^n U(x_i)$. Da $I$ eine gerichtete Menge ist, existiert ein $\beta \in I$ mit $\beta \succcurlyeq \alpha_{x_i}$ für alle $i \in \menge{1,\dots,n}$. Dann gilt $x_\beta \notin U(x_i)$ für alle $i \in \menge{1,\dots,n}$ im Widerspruch zur Überdeckung von $M$ durch die $U(x_i)$.
			\rueckrichtung Angenommen jedes Netz in $M$ habe einen Häufungspunkt und $M$ besitzt eine offene Überdeckung $\mathcal{O}$, die keine endliche Teilüberdeckung besitzt. Definieren wir uns nun eine gerichtete Menge $I$ als Menge aller endliche Teilüberdeckungen von $\mathcal{O}$ mit der Richtung $U \preccurlyeq V \defequiv U \subseteq V$. Für alle $A \in I$ wird $M$ nicht von $\bigcup_{U \in A} U$ überdeckt, d.h. wir können für alle $A \in I$ einen Punkt $x_A \in M \setminus \bigcup_{U \in A} U$ auswählen. Dies definiert uns wieder ein Netz $\folge{x_A}{A \in I}$. Dann hat $\folge{x_A}{A \in I}$ einen Häufungspunkt $x \in M$. Da $\mathcal{O}$ die Menge $M$ überdeckt, existiert ein $V \in \mathcal{O}$ mit $x \in V$ und es gilt $\menge{V} \in I$. Also existiert ein $A \succcurlyeq \menge{V}$, sodass $x_A \in V$. Das bedeutet jedoch, dass $A$ eine endliche Teilüberdeckung von $\mathcal{O}$ ist, die $V$ enthält im Widerspruch zur Wahl von $x_A$.
		\end{proof-equivalence}
		%
		\item \begin{proof-equivalence}
			\hinrichtung Angenommen $\abb{F}{X}{Y}$ sei stetig und $\folge{u_\alpha}{\alpha \in I} \subseteq X$ sei ein gegen $u \in X$ konvergentes Netz. Nehmen wir uns eine Umgebung $U \subseteq Y$ von $F(u)$, dann ist das Urbild $F^{-1}(U) \subseteq X$ eine Umgebung von $u$. Da $u_\alpha \to  x$ konvergiert, existiert ein $\alpha_0 \in I$, sodass $u_\alpha \in F^{-1}(U)$ für alle $\alpha \succcurlyeq \alpha_0$. Dies bedeutet wiederum, dass $F(u_\alpha) \in U$ für alle $\alpha \succcurlyeq \alpha_0$. Damit konvergiert auch $\folge{F(u_\alpha)}{\alpha \in I}$ gegen $F(u)$. 
			\rueckrichtung Betrachten wir nun erneut als gerichtete Menge $I = \mathcal{U}(u)$ die Menge aller Umgebungen von $u$ mit Richtung $U \preccurlyeq V \defequiv V \subseteq U$. 
			Nehmen wir nun an, dass $\abb{F}{X}{Y}$ nicht stetig sei, d.h. es existiert eine offene Menge $U \subseteq Y$, für die $F^{-1}(U)$ nicht offen ist. Dies bedeutet, dass $F^{-1}(U)$ einen Punkt $u$ enthält, für den jede Umgebung $V \subseteq X$ einen Netzpunkt $u_V \notin F^{-1}(U)$ enthält. Die gewählten Punkte $u_V$ bilden dann ein Netz $\folge{u_V}{V \in I}$. Dieses konvergiert gegen $u$, da für jede offene Umgebung $V \subseteq X$ von $u$ und ein beliebiges $V' \succcurlyeq V$ gilt, dass $u_{V'} \in V' \subseteq V$.
			Konvergiere nun das Netz $\folge{F(u_V)}{V \in I} \subseteq Y$ gegen $F(u)$. Ist $U \subseteq Y$ eine offene Umgebung von $F(u)$, dann existiert ein $V_0 \in I$, sodass $F(x_V) \in U$ für alle $V \succcurlyeq V_0$. Das bedeutet jedoch auch, dass $x_V \in F^{-1}(U)$ im Widerspruch zur Annahme.
		\end{proof-equivalence}
	\end{enumerate}

	\begin{exercise}
		Es sei $\Omega$ ein kompakter topologischer Raum. Beweisen Sie:
		\begin{enumerate}
			\item Ist $\Omega$ ein metrischer Raum, so ist $\Omega$ vollständig.
			\item Der Raum $\mathcal{C}(\Omega,\Rm)$ der stetigen Funktionen $\abb{u}{\Omega}{\Rm}$ ausgestattet mit der kanonischen Metrik
			\begin{equation*}
				d(u,v) \defeq \max_{x \in \Omega} \abs{u(x) - v(x)}
			\end{equation*}
			ist vollständig.
		\end{enumerate}
	\end{exercise}
	\begin{enumerate}[leftmargin=\zulength, label=(zu \alph*)]
		\item Sei $\folge{\omega_n}{n \in \N} \subseteq \Omega$ eine Cauchy-Folge. Da $\Omega$ kompakt und metrisch ist, also insbesondere auch folgenkompakt, existiert eine konvergente Teilfolge $\folge{\omega_{n_k}}{k}$ mit $\omega_{n_k} \to \omega \in \Omega$. Daher existiert ein $N_1$, sodass $d(\omega_{n_k},\omega) < \sfrac{\epsilon}{2}$ für alle $n \ge N_1$. Sei $N_2$ so gewählt, dass für $n,m \ge N_2$ gilt $d(\omega_n,\omega_m) < \sfrac{\epsilon}{2}$. Für $n \ge N \defeq \max\menge{N_1,N_2}$ gilt dann nach Dreiecksungleichung 
		\begin{equation*}
			d(\omega_n,\omega) \le d(\omega_n,\omega_N) + d(\omega_n,\omega) < \epsilon
		\end{equation*}
		Also gilt $\omega_n \to \omega$ und $\Omega$ ist vollständig.
	\end{enumerate}

	%%%% AUFGABE 9 %%%%
	\begin{exercise}
		Es seien $(X, \norm{\cdot}_X)$ und $(Y, \norm{\cdot}_Y)$ normierte Räume über $\K$. Beweisen Sie:
		\begin{enumerate}
			\item Jede Cauchyfolge in X ist beschränkt.
			\item $\abs{\cdot}$ sei eine $p$-Norm auf $\K^2$ ($1 \le p \le \infty)$. Dann ist
			\begin{equation*}
				\bigabb{\norm{\ \cdot \ }}{X \times Y}{\K}{(u,v)}{\norm{(u,v)} \defeq \abs{(\norm{u}_X , \norm{u}_Y)}}
			\end{equation*}
			eine Norm auf dem linearen Raum $X \times Y$.
			\item $X \times X$ sei mit der Norm $\norm{(u, v)} = \norm{u}_X + \norm{v}_X$ ausgestattet und $\K \times X$ mit der Norm $\norm{(\alpha,u)} = \abs{\alpha} + \norm{u}_X$. Dann sind Addition, Skalarmultiplikation und Norm stetige Abbildungen.
		\end{enumerate}
	\end{exercise}

	\begin{enumerate}[leftmargin=\zulength, label=(zu \alph*)]
		\item Sei $\folge{x_n}{n \in \N} \subseteq X$ eine Cauchy-Folge, d.h. $\norm{x_n - x_m} < \epsilon$ für alle $n,m \ge N$ für ein $N \in \N$. Mithilfe der Dreiecksungleichung folgt
		\begin{equation*}
			\norm{x_n} - \norm{x_m} \le \norm{x_n - x_m} < \epsilon \follows \norm{x_n} < \epsilon + \norm{x_m} \qquad \forall n,m \ge N
		\end{equation*}
		Setze nun $m = N$, dann ist $\norm{x_n} < \norm{x_N} + \epsilon$ für alle $n \ge N$, d.h.
		\begin{equation*}
			\norm{x_n} < \max \menge{\norm{x_1}, \norm{x_2}, \dots , \norm{x_{N-1}}, \norm{x_N} + \epsilon} \defqe M
		\end{equation*} 
		für alle $n \in \N$. Also ist $M$ eine Schranke für $\folge{x_n}{n \in \N}$.
		%
		\item  
		\begin{enumerate}[label=(\textit{\roman*})]
			\item Ist $(u,v) = 0$, dann gilt $\norm{(u,v)} = \abs{(\norm{0}_X, \norm{0}_Y)}_p = \abs{(0,0)}_p = 0$.
			Sei umgekehrt $0 = \norm{(u,v)} = \abs{(\norm{u}_X , \norm{v}_Y)}_p$. Dann ist $(\norm{u}_X , \norm{v}_Y) = 0$ (Normeigenschaft von $\abs{\ \cdot \ }_p$) und da auch $\norm{\ \cdot \ }_X$ und $\norm{\ \cdot \ }_Y$ Normen sind, folgt daraus $u = 0 = v$.
			\item  Sei $\lambda \in \K$. Dann gilt 
			\begin{align*}
				\norm{\lambda (u,v)} = \norm{(\lambda u , \lambda  v)} 
				&= \abs{(\norm{\lambda u}_X , \norm{\lambda v}_Y)}_p \\
				&= \abs{(\abs{\lambda} * \norm{u}_X , \abs{\lambda} * \norm{v}_Y)}_p \\
				&= \abs{\abs{\lambda} * (\norm{u}_X , \norm{v}_Y)}_p \\
				&= \abs{\lambda} * \abs{(\norm{u}_X , \norm{v}_Y)}_p \\
				&= \abs{\lambda} * \norm{(u,v)}
			\end{align*}
			\pagebreak
			\item Seien $(u,v),(u',v') \in X \times Y$. Mit $\triangle_X$ bezeichnen wir die Anwendung der Dreiecksungleichung von Norm $X$.
			\begin{align*}
				\norm{(u,v) + (u',v')} &= \norm{(u+u',v+v')} \\
				&= \abs{(\norm{u+u'}_X, \norm{v+v'}_Y)}_p \\
				&\le \abs{(\norm{u}_X + \norm{u'}_X , \norm{v}_Y + \norm{v'}_Y)}_p & (\triangle_X, \triangle_Y) \\
				&= \abs{(\norm{u}_X, \norm{v}_Y) + (\norm{u'}_X, \norm{v'}_Y)}_p \\
				&\le \abs{(\norm{u}_X , \norm{v}_Y)}_p + \abs{(\norm{u'}_X , \norm{v'}_Y)}_p & (\triangle_p) \\
				&= \norm{(u,v)} + \norm{(u',v')}
			\end{align*}
		\end{enumerate}
		
%		Sei $p = \infty$. Dann ist $\norm{(u,v)} = \max\menge{\norm{u}_X, \norm{v}_Y}$. 
%		\begin{enumerate}[label=(\textit{\roman*})]
%			\item Ist $(u,v) = 0$, dann gilt $\norm{(u,v)} = \max\menge{\norm{0}_X, \norm{0}_Y} = 0$. 
%			Sei umgekehrt $0 = \norm{(u,v)} = \max\menge{\norm{u}_X, \norm{v}_Y}$. Da $\norm{u}_X , \norm{v}_Y \ge 0$, muss schon $\norm{u}_X = 0$ und $\norm{v}_Y = 0$ folgen. Da $\norm{\cdot}_X$ und $\norm{\cdot}_Y$ Normen sind, ist schließlich $u = 0 = v$.
%			\item  Sei $\lambda \in \K$, $u \in X$ und $v \in Y$. Dann gilt 
%			\begin{equation*}
%			\begin{aligned}
%			\norm{\lambda (u,v)} = \norm{(\lambda u , \lambda  v)} &= \max\menge{\norm{\lambda u}_X , \norm{\lambda v}_Y} \\
%			&= \max\menge{\abs{\lambda} * \norm{u}_X , \abs{\lambda} * \norm{v}_Y} \\
%			&= \abs{\lambda} * \max\menge{\norm{u}_X, \norm{v}_Y} \\
%			&= \abs{\lambda} * \norm{(u,v)}
%			\end{aligned}
%			\end{equation*}
%			\item Sei $u_1,u_2 \in X$ und $v_1,v_2 \in Y$.
%			\begin{align*}
%				\norm{(u_1,v_1) + (u_2,v_2)} &= \norm{(u_1 + u_2, v_1 + v_2)} \\
%				&= \max\menge{\norm{u_1+u_2}_X, \norm{v_1 + v_2}_Y} \\
%				&\le \max\menge{\norm{u_1}_X + \norm{u_2}_X , \norm{v_1}_Y + \norm{v_2}_Y} \\
%				&\le \max\menge{\norm{u_1}_X + \norm{v_1}_Y} + \max\menge{\norm{u_2}_X + \norm{v_2}_Y} \\
%				&= \norm{(u_1,v_1)} + \norm{(u_2,v_2)}
%			\end{align*}
%		\end{enumerate}
	\end{enumerate}
\end{exercisePage}